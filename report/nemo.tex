
%formato de plantilla que se va a utilizar
\documentclass[a4paper]{article} 
%para idioma espanol
\usepackage[utf8]{inputenc}
\usepackage[spanish]{babel}
%gestor de espacio
\usepackage[margin=2cm,top=2cm,includefoot]{geometry}
%gestor de imagenes
\usepackage{graphicx}
%usar el float H
\usepackage{float}
%deteccion de color
\usepackage[table,xcdraw]{xcolor}
%insericion de cuadros
\usepackage[most]{tcolorbox}
%definir el estilo de lapagina
\usepackage{fancyhdr}
%gestion de hypervinculos
\usepackage[hidelinks]{hyperref}
%eliminar sangria inicial
\usepackage{parskip}
%cabecera
\setlength{\headheight}{40pt}
\pagestyle{fancy}
\fancyhf{}
\lhead{\includegraphics[width=1cm]{\logoWhip}}
\rhead{\includegraphics[height=1cm]{\logoDNS}}
\renewcommand{\headrulewidth}{3pt}
\renewcommand{\headrule}{\hbox to \headwidth{\color{lineCabecera}\leaders\hrule height \headrulewidth\hfill}}

%variables de color
\definecolor{greenPortada}{HTML}{69A84F}
\definecolor{lineCabecera}{HTML}{5DADE2}


%variables globales
\newcommand{\canal}{Nemotek}
\newcommand{\startDate}{\today}

\newcommand{\logoDNS}{img/largeDns.pdf}
\newcommand{\logoWhip}{img/largeW.png}
\newcommand{\first}{img/overView.png}
\newcommand{\second}{img/metricaTotal.png}
\newcommand{\cra}{img/client-request-allow.png}
\newcommand{\crb}{img/client-request-block.png}
\newcommand{\cta}{img/client-threat-allow.png}
\newcommand{\ctb}{img/client-threat-block.png}
\newcommand{\daa}{img/domain-all-all.png}
\newcommand{\dta}{img/domain-threat-allow.png}
\newcommand{\dtb}{img/domain-threat-block.png}
\newcommand{\dtall}{img/domain-threat-all.png}
\newcommand{\tbc}{/img/threats-by-category.pdf}
\newcommand{\tbd}{/img/threats-by-domain.pdf}
\newcommand{\tbr}{/img/threats-by-client.pdf}
\newcommand{\tbs}{/img/threats-by-site.pdf}
\newcommand{\tbu}{/img/threats-by-user.pdf}
% \newcommand{\}{/img/}
% \newcommand{\}{/img/}


%comienzo del documento
\begin{document}
    %creacion de portada
    \begin{titlepage}
        \centering
        \includegraphics[width=0.2\textwidth]{\logoWhip}
        \par\vspace{1cm}
        {\LARGE\textbf{Informe tecnico}}
        \par\vspace{0.4cm}
        {\Huge\bfseries\textcolor{greenPortada}{\canal}}
        \vfill

        \par\vspace{0.4cm}
        \includegraphics[width=\textwidth,height=10cm,keepaspectratio]{\logoDNS}
        \vfill

        \par\vspace{1cm}
        \begin{tcolorbox}[colback=red!5!white,colframe=red!75!black]
            \centering
            Este documento es confidencial y puede contener información sensible
            \\
            No debería ser compartido con terceras entidades
        \end{tcolorbox}
        \vfill

        {\large\startDate\par}

    \end{titlepage}
    \clearpage

    %desarrollo
    \section{Resultados de la prueba de concepto}
    Whip Solutions ha llevado la prueba de concepto de DNS Filter en conjunto con Nemotek del día 28 de junio del 2024 al día 21 de julio del 2024, tras este proceso de PoC podemos visualizar la siguiente información en la plataforma de DNS Filter.

    \begin{figure}[H] 
        \centering 
        \includegraphics[width=0.9 \linewidth]{\first} 
        \caption{Vista General} 
    \end{figure} 

    Se puede observar el total de peticiones, peticiones permitidas, peticiones bloqueadas y las amenazas que se tuvieron durante dicho proceso, la prueba se llevó a cabo con 3 clientes que se instalaron sobre sus dispositivos, se desplegaron diversas políticas con las cuales se probaron las funciones de DNS Filter.
    Se configuró solo un sitio y de este se pueden obtener las siguientes métricas. 

    \begin{figure}[H] 
        \centering 
        \includegraphics[width=0.9 \linewidth]{\second} 
        \caption{Métricas Totales} 
    \end{figure} 

    Podemos visualizar el top de 5 DNS más consultados en todo nuestro tráfico además de las categorías a las que pertenecen, las amenazas que fueron bloqueadas, las que fueron permitidas, se puede observar los clientes que realizaron más peticiones permitidas, de amenazas y que fueron bloqueadas, también los usuarios que realizaron todas estas consultas con esto podemos monitorizar el uso de los dispositivos y de la red que estamos monitoreando, permitiéndose una vista completa de todas las consultas hechas.

    \begin{figure}[H] 
        \centering 
        \includegraphics[width=0.9 \linewidth]{\daa} 
        \caption{Top de DNS de las peticiones totales y categorías} 
    \end{figure}

    \begin{figure}[H]
        \centering
        \includegraphics[width=0.9 \linewidth]{\dta}
        \caption{Top de DNS de las peticiones maliciosas permitidas}
    \end{figure}

    \begin{figure}[H]
        \centering
        \includegraphics[width=0.9 \linewidth]{\dtb}
        \caption{Top de DNS de las peticiones maliciosas denegadas}
    \end{figure}

    \begin{figure}[H]
        \centering
        \includegraphics[width=0.9 \linewidth]{\cra}
        \caption{Top de Roaming client con peticiones permitidas}
    \end{figure}

    \begin{figure}[H]
        \centering
        \includegraphics[width=0.9 \linewidth]{\crb}
        \caption{Top de Roaming client con peticiones denegadas}
    \end{figure}

    \begin{figure}[H]
        \centering
        \includegraphics[width=0.9 \linewidth]{\cta}
        \caption{Top de Roaming client con peticiones maliciosas permitidas}
    \end{figure}

    \begin{figure}[H]
        \centering
        \includegraphics[width=0.9 \linewidth]{\ctb}
        \caption{Top de Roaming client con peticiones maliciosas denegadas}
    \end{figure}

    \begin{figure}[H]
        \centering
        \includegraphics[width=0.9 \linewidth]{\tbc}
        \caption{Amenazas por categoría}
    \end{figure}

    \begin{figure}[H]
        \centering
        \includegraphics[width=0.9 \linewidth]{\tbr}
        \caption{Amenazas por cliente}
    \end{figure}   

    \begin{figure}[H]
        \centering
        \includegraphics[width=0.9 \linewidth]{\tbd}
        \caption{Amenazas por dominio}
    \end{figure}    
    
    \begin{figure}[H]
        \centering
        \includegraphics[width=0.9 \linewidth]{\tbs}
        \caption{Amenazas por sitio}
    \end{figure}

    \begin{figure}[H]
        \centering
        \includegraphics[width=0.9 \linewidth]{\tbu}
        \caption{Amenazas por usuario}
    \end{figure}
    
    
    
    
    
    
\end{document}

