%formato de plantilla que se va a utilizar
\documentclass[a4paper]{article} 
%para idioma espanol
\usepackage[utf8]{inputenc}
\usepackage[spanish]{babel}
%gestor de espacio
\usepackage[margin=2cm,top=2cm,includefoot]{geometry}
%gestor de imagenes
\usepackage{graphicx}
%usar el float H
\usepackage{float}
%deteccion de color
\usepackage[table,xcdraw]{xcolor}
%insericion de cuadros
\usepackage[most]{tcolorbox}
%definir el estilo de la pagina
\usepackage{fancyhdr}
%gestion de hypervinculos
\usepackage[hidelinks]{hyperref}
%eliminar sangria inicial
\usepackage{parskip}

%cabecera
\setlength{\headheight}{40pt}
\pagestyle{fancy}
\fancyhf{}
\lhead{\includegraphics[width=1cm]{\logoWhip}}
\rhead{\includegraphics[height=1cm]{\logoMarca}}
\renewcommand{\headrulewidth}{3pt}
\renewcommand{\headrule}{\hbox to \headwidth{\color{lineTeramind}\leaders\hrule height \headrulewidth\hfill}}

%variables de color
\definecolor{greenPortada}{HTML}{69A84F}
\definecolor{lineCabecera}{HTML}{5DADE2}
\definecolor{lineTeramind}{HTML}{F39C12}



%variables globales
\newcommand{\canal}{AMEZQUITA}
\newcommand{\startDate}{\today}

\newcommand{\logoMarca}{img/teramindComplete.png}
\newcommand{\logoWhip}{img/largeW.png}
\newcommand{\first}{img/captura1.png}
\newcommand{\second}{img/captura2.png}
\newcommand{\thirdth}{img/captura3.png}
\newcommand{\fourth}{img/captura4.png}
\newcommand{\fiveth}{img/captura5.png}



% \newcommand{\}{/img/}


%comienzo del documento
\begin{document}
    %creacion de portada
    \begin{titlepage}
        \centering
        \includegraphics[width=0.2\textwidth]{\logoWhip}
        \par\vspace{1cm}
        {\LARGE\textbf{Información Tecnica}}
        \par\vspace{0.9cm}
        {\Huge\bfseries\textcolor{greenPortada}{\canal}}
        \vfill

        \par\vspace{0.4cm}
        \includegraphics[width=0.8\textwidth,height=5cm,keepaspectratio]{\logoMarca}
        \vfill

        \par\vspace{1cm}
        % \begin{tcolorbox}[colback=red!5!white,colframe=red!75!black]
        %     \centering
        %     Este documento es confidencial y puede contener información sensible
        %     \\
        %     No debería ser compartido con terceras entidades
        % \end{tcolorbox}
        % \vfill

        {\large\startDate\par}

    \end{titlepage}
    \clearpage

    %desarrollo
    Información para crear regla de seguimiento en falsificación de actividad.
    Se crearon 2 reglas para poder disparar alarmas de que alguien incurra en estas prácticas.
    Como dato general ambas están en la categoría de actividad y pulsación de teclado.

    \begin{figure}[H] 
        \centering 
        \includegraphics[width=0.9 \linewidth]{\first} 
        \caption{General} 
    \end{figure} 
    
    \section{Regla 1}
    Se crea una regla para que se reporte cualquier carácter repetido, esto se hace a través del uso de expresiones regulares, tomando la siguiente sentencia.
    
    /(.)\1{9,}/



    \section{Regla 2}

    
\end{document}

